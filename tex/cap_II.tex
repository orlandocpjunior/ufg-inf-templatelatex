\chapter{Descrição da classe \textsf{inf-ufg}}
\label{cap:theory}

Este capítulo apresentará uma visão geraL...

%% - - - - - - - - - - - - - - - - - - - - - - - - - - - - - - - - - - -
\section{Opções da classe}
\label{sec:http2}

Para usar esta classe num documento \LaTeXe, coloque os arquivos 
\verb|inf-ufg.cls|,\ \verb|inf-ufg.bst|,\ \verb|abnt-num.bst|,\ \verb|atbeginend.sty|\ e \verb|tocloft.sty|\ numa pasta onde o compilador \LaTeX\ pode achá--lo (normalmente na mesma pasta que seu arquivo \verb|.tex|), e defina--o como o estilo do seu documento. Por exemplo, uma dissertação de mestrado que usa o modelo abnt de citações bibliográficas

\subsection{Sub seção}
\label{subsec:framing}

As opções da classe são \verb|[tese]| (para tese de doutorado), \verb|[dissertacao]| (para dissertação de mestrado), \verb|[monografia]| (para monografia de curso de especialização e \verb|[relatorio]| (para relatório final de curso de graduação). Se nenhuma opção for declarada, o documento é considerado como uma dissertação de mestrado. Se a opção \verb|[abnt]| for utilizada, as citações bibliográficas serão geradas conforme definido pelo grupo de trabalho \textsf{abnt-tex}. Contudo, o mais recomendável é não utilizar essa opção. Com a opção \verb|[nocolorlinks]| todos os {\em links} de navegação no texto ficam na cor preta. O ideal é usar esta opção para gerar o arquivo para impressão, pois a qualidade da impressão dos {\em links} fica superior.

%% - - - - - - - - - - - - - - - - - - - - - - - - - - - - - - - - - - -

\section{Parâmetros da classe}
\label{sec:lua}

Os elementos pré-textuais são definidos página por página e dependem da correta definição dos parâmetros listados a seguir (aqueles que contém um texto/valor padrão não precisam ser definidos, caso atenda a situação do autor do texto que está usando a classe
